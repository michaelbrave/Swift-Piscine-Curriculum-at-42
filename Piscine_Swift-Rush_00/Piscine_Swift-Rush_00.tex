%%%%%%%%%%%%  Generated using docx2latex.com  %%%%%%%%%%%%%%

%%%%%%%%%%%%  v2.0.0-beta  %%%%%%%%%%%%%%

\documentclass[12pt]{report}
\usepackage{amsmath}
\usepackage{latexsym}
\usepackage{amsfonts}
\usepackage[normalem]{ulem}
\usepackage{array}
\usepackage{amssymb}
\usepackage{graphicx}
\usepackage[backend=biber,
style=numeric,
sorting=none,
isbn=false,
doi=false,
url=false,
]{biblatex}\addbibresource{bibliography.bib}

\usepackage{subfig}
\usepackage{wrapfig}
\usepackage{wasysym}
\usepackage{enumitem}
\usepackage{adjustbox}
\usepackage{ragged2e}
\usepackage[svgnames,table]{xcolor}
\usepackage{tikz}
\usepackage{longtable}
\usepackage{changepage}
\usepackage{setspace}
\usepackage{hhline}
\usepackage{multicol}
\usepackage{tabto}
\usepackage{float}
\usepackage{multirow}
\usepackage{makecell}
\usepackage{fancyhdr}
\usepackage[toc,page]{appendix}
\usepackage[hidelinks]{hyperref}
\usetikzlibrary{shapes.symbols,shapes.geometric,shadows,arrows.meta}
\tikzset{>={Latex[width=1.5mm,length=2mm]}}
\usepackage{flowchart}\usepackage[paperheight=11.0in,paperwidth=8.5in,left=0.5in,right=0.5in,top=0.5in,bottom=0.5in,headheight=1in]{geometry}
\usepackage[utf8]{inputenc}
\usepackage[T1]{fontenc}
\TabPositions{0.5in,1.0in,1.5in,2.0in,2.5in,3.0in,3.5in,4.0in,4.5in,5.0in,5.5in,6.0in,6.5in,7.0in,}

\urlstyle{same}


 %%%%%%%%%%%%  Set Depths for Sections  %%%%%%%%%%%%%%

% 1) Section
% 1.1) SubSection
% 1.1.1) SubSubSection
% 1.1.1.1) Paragraph
% 1.1.1.1.1) Subparagraph


\setcounter{tocdepth}{5}
\setcounter{secnumdepth}{5}


 %%%%%%%%%%%%  Set Depths for Nested Lists created by \begin{enumerate}  %%%%%%%%%%%%%%


\setlistdepth{9}
\renewlist{enumerate}{enumerate}{9}
		\setlist[enumerate,1]{label=\arabic*)}
		\setlist[enumerate,2]{label=\alph*)}
		\setlist[enumerate,3]{label=(\roman*)}
		\setlist[enumerate,4]{label=(\arabic*)}
		\setlist[enumerate,5]{label=(\Alph*)}
		\setlist[enumerate,6]{label=(\Roman*)}
		\setlist[enumerate,7]{label=\arabic*}
		\setlist[enumerate,8]{label=\alph*}
		\setlist[enumerate,9]{label=\roman*}

\renewlist{itemize}{itemize}{9}
		\setlist[itemize]{label=$\cdot$}
		\setlist[itemize,1]{label=\textbullet}
		\setlist[itemize,2]{label=$\circ$}
		\setlist[itemize,3]{label=$\ast$}
		\setlist[itemize,4]{label=$\dagger$}
		\setlist[itemize,5]{label=$\triangleright$}
		\setlist[itemize,6]{label=$\bigstar$}
		\setlist[itemize,7]{label=$\blacklozenge$}
		\setlist[itemize,8]{label=$\prime$}

\setlength{\topsep}{0pt}\setlength{\parindent}{0pt}

 %%%%%%%%%%%%  This sets linespacing (verticle gap between Lines) Default=1 %%%%%%%%%%%%%%


\renewcommand{\arraystretch}{1.3}


%%%%%%%%%%%%%%%%%%%% Document code starts here %%%%%%%%%%%%%%%%%%%%



\begin{document}

\vspace{\baselineskip}

\vspace{\baselineskip}

\vspace{\baselineskip}

\vspace{\baselineskip}

\vspace{\baselineskip}
\par

\section*{My Media Library}
\addcontentsline{toc}{section}{My Media Library}

\vspace{\baselineskip}

\vspace{\baselineskip}
\paragraph*{Michael BRAVE }
\addcontentsline{toc}{paragraph}{Michael BRAVE }
\paragraph*{42 Staff }
\addcontentsline{toc}{paragraph}{42 Staff }

\vspace{\baselineskip}

\vspace{\baselineskip}

\vspace{\baselineskip}

\vspace{\baselineskip}
\begin{Center}
\textit{Summary: This document contains the subject for Day for the $``$Piscine Swift$"$  from 42}
\end{Center}\par


\vspace{\baselineskip}

\vspace{\baselineskip}

\vspace{\baselineskip}

\vspace{\baselineskip}

\vspace{\baselineskip}

\vspace{\baselineskip}

\vspace{\baselineskip}

\vspace{\baselineskip}


 %%%%%%%%%%%%  Starting New Page here %%%%%%%%%%%%%%

\newpage

\vspace{\baselineskip}
\vspace{\baselineskip}
\section*{Contents}
\addcontentsline{toc}{section}{Contents}

\vspace{\baselineskip}
\subsubsection*{I\hspace*{10pt}\hspace*{10pt}Foreword}
\addcontentsline{toc}{subsubsection}{I\hspace*{10pt}\hspace*{10pt}Foreword}
\subsubsection*{II\hspace*{10pt}\hspace*{10pt}General Instructions}
\addcontentsline{toc}{subsubsection}{II\hspace*{10pt}\hspace*{10pt}General Instructions}
\subsubsection*{III\hspace*{10pt}\hspace*{10pt}Introduction}
\addcontentsline{toc}{subsubsection}{III\hspace*{10pt}\hspace*{10pt}Introduction}
\subsubsection*{IV\hspace*{10pt}\hspace*{10pt}Exercise 00: My Media Library}
\addcontentsline{toc}{subsubsection}{IV\hspace*{10pt}\hspace*{10pt}Exercise 00: My Media Library}
\subsubsection*{V\hspace*{10pt}\hspace*{10pt}Bonus: }
\addcontentsline{toc}{subsubsection}{V\hspace*{10pt}\hspace*{10pt}Bonus: }

\vspace{\baselineskip}

\vspace{\baselineskip}

\vspace{\baselineskip}

\vspace{\baselineskip}

\vspace{\baselineskip}

\vspace{\baselineskip}


 %%%%%%%%%%%%  Starting New Page here %%%%%%%%%%%%%%

\newpage

\vspace{\baselineskip}
\vspace{\baselineskip}
\section*{Chapter I}
\addcontentsline{toc}{section}{Chapter I}
\section*{Foreword}
\addcontentsline{toc}{section}{Foreword}

\vspace{\baselineskip}
Kodi (formerly XBMC) is worthy of study if for no other reason because it is a long running open source project that has significant household name recognition. From their own wiki on the History of Kodi. \par


\vspace{\baselineskip}
\subsubsection*{Xbox Media Player}
\addcontentsline{toc}{subsubsection}{Xbox Media Player}
Xbox Media Player, (or XBMP for short), was the predecessor to XBMC a feature-rich free and open source media player for the Xbox, licensed under the GNU General Public License (GPL). With an audio/video-player-core based on MPlayer, it allowed owners of a modified Xbox to display pictures and movie files, as well as play music files from the Xbox DVD-ROM drive, built-in harddisk drive, LAN (using SMB) or the Internet.\par


\vspace{\baselineskip}
The Xbox Media Player Project was founded by d7o3g4q (also known as duo) and RUNTiME. It started out as two separate players, with the two developers each working on their own design and code. After sharing code and coordinating features to not duplicate efforts, by XBoxMediaPlayer beta 5 the two players were merged. The development and beta-testing was done "behind closed doors" for this project (d7o3g4q and RUNTiME promising that when version 1.0 was made they would release the source code to the public). After beta 6 was completed there were complaints from a lot of people as to why the developers did not release the source code for the player sooner as they were using FFmpeg and Xvid code which are under the (L)GPL license. Even though the project was closed, d7o3g4q and RUNTiME released the source code for beta 6 on October 15, 2002.\par


\vspace{\baselineskip}
In November 2002, another software developer nicknamed Frodo who was the founder of "YAMP - Yet Another Media Player" joined the Xbox Media Player team and the XBoxMediaPlayer and YAMP projects were merged. The first release of the merged projects was called "Xbox Media Player 2.0" and the source code for it was released on December 14, 2002. XBoxMediaPlayer 2.0 was a complete rewrite using a new core based on the MPlayer project, still using FFmpeg/XVID codec code. On December 28, 2002, the source code of XBoxMediaPlayer 2.1 was released with many bug fixes and a couple of new features such as true AC3 5.1 output, volume normalizer/amplification and an additional post processing filter. Two weeks later on January 12, 2003, XBoxMediaPlayer 2.2 source code was released with new features including dashboard mode to launch other Xbox applications/games, separate national language files, streaming media from windows file shares (SMB), audio-playlist, the ability to play media on-the-fly from ISO9660-Mode1 CDs and Windows DLL support for WMV 7,8, and 9. Xbox Media Player development stopped on December 13, 2003.\par


\vspace{\baselineskip}
\subsubsection*{Xbox Media Center}
\addcontentsline{toc}{subsubsection}{Xbox Media Center}
Xbox Media Player development stopped on 13 December, 2003, by which time its successor, Xbox Media Center, was ready for its debut, renamed as it was growing out of its 'player' name and into a 'center' for media playback. The first stable release of XBMC was on 29 June, 2004, with the official release of XboxMediaCenter 1.0.0. This announcement also encouraged everyone using XBMP or XBMC Beta release to update, as all support for those previous versions would be dropped, and they would only officially support version 1.0.0. Not featured in XBMP, the addition of embedded Python was given the ability to draw interface elements in the GUI (graphical user interface), and allowed user and community generated scripts to be executed within the XBMC environment.\par


\vspace{\baselineskip}
With the release of 1.0.0 in the middle of 2004, work continued on the XBMC project to add more features, such as support for iTunes features like DAAP and smart playlists, as well as lots of improvements and fixes. The second stable release of XBMC, 1.1.0, was released on 18 October, 2004. This release included support for more media types, file types, container formats, as well as video playback of Nullsoft streaming videos and karaoke support (CD-G).\par


\vspace{\baselineskip}
After two years of heavy development, XBMC announced a stable point final release of XBMC 2.0.0 on 29 September, 2006. Even more features were packed into the new version with the addition of RAR and ZIP archive support, a brand new player interface with support for multiple players. Such players include PAPlayer, the new audio/music player with crossfade, gapless playback and ReplayGain support, and the new DVDPlayer with support for menu and navigation support as well as ISO/IMG image parsing. Prior to this point release, XBMC just used a modified fork of MPlayer for all of its media needs, so this was a big step forward. Support for iTunes 6.x DAAP, and UPnP clients for streaming was also added. A reworked skinning engine was included in this release to provide a more powerful way to change the appearance of XBMC. The last two features include read-only support for FAT12/16/32 formatted USB Mass Storage devices, and a "skinnable" 3D visualizer.\par


\vspace{\baselineskip}
The release of XBMC 2.0.1 on 12 November 2006 contained numerous bug fixes that made it through the 2.0.0 release. This also marked the change from CVS to SVN (Subversion) for the development tree.\par


\vspace{\baselineskip}
On 29 May 2007, the team behind XBMC put out a call for developers interested in porting XBMC to the Linux operating system. Since a few developers on Team XBMC had already begun porting parts of XBMC over to Linux using SDL and OpenGL as a replacement for DirectX, which XBMC was using heavily on the Xbox version of XBMC.\par


\vspace{\baselineskip}
Version numbers 8.10, 9.04, and 9.11 reflected the release year and month. This versioning system was abandoned for v10.\par


\vspace{\baselineskip}
\subsubsection*{XBMC Media Center}
\addcontentsline{toc}{subsubsection}{XBMC Media Center}
Sometime in 2009 Xbox Media Center's name was changed to just XBMC Media Center™.\par


\vspace{\baselineskip}
On 27 May, 2010, Team XBMC announced the splitting of the Xbox branch into a new project; XBMC4Xbox which will continue the development and support of XBMC for the old Xbox hardware platform as a separate project, with the original XBMC project no longer offering any support for the Xbox.\par


\vspace{\baselineskip}
Kodi\par

Later on August 2014, an announcement was made of release of version 14 that the name would change to Kodi.\par


\vspace{\baselineskip}


 %%%%%%%%%%%%  Starting New Page here %%%%%%%%%%%%%%

\newpage

\vspace{\baselineskip}
\vspace{\baselineskip}
\section*{Chapter II}
\addcontentsline{toc}{section}{Chapter II}
\section*{General Instructions}
\addcontentsline{toc}{section}{General Instructions}
\begin{itemize}
	\item Only this document will serve as reference. Do not trust rumors.\par

	\item Read carefully the whole subject before beginning.\par

	\item Watch out! This document could potentially change up to an hour before submission.\par

	\item This project will be corrected by humans only.\par

	\item This course is designed to build on previous days’ concepts, try your hardest to finish everyday.\par

	\item Each day culminates in a portfolio piece, if you finish the day this is something you can use to get hired.\par

	\item When submitting, submit the folder of the Xcode project.\par

	\item Only the work submitted on the repository will be accounted for during peer-2-peer correction.\par

	\item Here it is the \href{https://docs.swift.org/swift-book/}{\textcolor[HTML]{1155CC}{\uline{official manual of Swift}}} and the \href{https://developer.apple.com/documentation/swift/swift_standard_library}{\textcolor[HTML]{1155CC}{\uline{Swift Standard Library}}}\par

	\item It is forbidden to use other libraries, packages, pods, etc. Unless otherwise stated in the project.\par

	\item Got a question? Ask your peer on the right. Otherwise, try your peer on the left.\par

	\item You can discuss on the Piscine forum of your Intra!\par

	\item By Odin, by Thor! Use your brain!!!
\end{itemize}\par


\vspace{\baselineskip}

\vspace{\baselineskip}


 %%%%%%%%%%%%  Starting New Page here %%%%%%%%%%%%%%

\newpage

\vspace{\baselineskip}
\vspace{\baselineskip}
\section*{Chapter III}
\addcontentsline{toc}{section}{Chapter III}
\section*{Introduction}
\addcontentsline{toc}{section}{Introduction}

\vspace{\baselineskip}

\vspace{\baselineskip}


 %%%%%%%%%%%%  Starting New Page here %%%%%%%%%%%%%%

\newpage

\vspace{\baselineskip}
\vspace{\baselineskip}
\section*{Chapter IV}
\addcontentsline{toc}{section}{Chapter IV}
\section*{Exercise 00 : My Media Library}
\addcontentsline{toc}{section}{Exercise 00 : My Media Library}

\vspace{\baselineskip}

\vspace{\baselineskip}

\vspace{\baselineskip}


%%%%%%%%%%%%%%%%%%%% Table No: 1 starts here %%%%%%%%%%%%%%%%%%%%


\begin{table}[H]
 			\centering
\begin{tabular}{p{7.3in}}
\hline
%row no:1
\multicolumn{1}{|p{7.3in}|}{\Centering Exercise : 00} \\
\hhline{-}
%row no:2
\multicolumn{1}{|p{7.3in}|}{\Centering My Media Library} \\
\hhline{-}
%row no:3
\multicolumn{1}{|p{7.3in}|}{Files to turn in: .xcodeproj and all necessary files} \\
\hhline{-}
%row no:4
\multicolumn{1}{|p{7.3in}|}{Allowed functions : Swift Standard Library, UIKit} \\
\hhline{-}
%row no:5
\multicolumn{1}{|p{7.3in}|}{Notes : n/a} \\
\hhline{-}

\end{tabular}
 \end{table}


%%%%%%%%%%%%%%%%%%%% Table No: 1 ends here %%%%%%%%%%%%%%%%%%%%


\vspace{\baselineskip}
We are creating a media library that will play movies from off of the internal storage of the device. This will search for files on device, display them by title, download metadata from IMDB, and list that metadata next to each movie. These should be searchable by genre, director and actor. There should also be a way to rate them, and display them by rating. \par


\vspace{\baselineskip}

\vspace{\baselineskip}


 %%%%%%%%%%%%  Starting New Page here %%%%%%%%%%%%%%

\newpage

\vspace{\baselineskip}
\vspace{\baselineskip}

\vspace{\baselineskip}

\vspace{\baselineskip}

\vspace{\baselineskip}
\section*{Chapter V}
\addcontentsline{toc}{section}{Chapter V}
\section*{Bonus : Other File Formats}
\addcontentsline{toc}{section}{Bonus : Other File Formats}

\vspace{\baselineskip}

\vspace{\baselineskip}

\vspace{\baselineskip}


%%%%%%%%%%%%%%%%%%%% Table No: 2 starts here %%%%%%%%%%%%%%%%%%%%


\begin{table}[H]
 			\centering
\begin{tabular}{p{7.3in}}
\hline
%row no:1
\multicolumn{1}{|p{7.3in}|}{\Centering Bonus} \\
\hhline{-}
%row no:2
\multicolumn{1}{|p{7.3in}|}{\Centering Other File Formats} \\
\hhline{-}
%row no:3
\multicolumn{1}{|p{7.3in}|}{Files to turn in: .xcodeproj and all necessary files} \\
\hhline{-}
%row no:4
\multicolumn{1}{|p{7.3in}|}{Allowed functions : Swift Standard Library, UIKit} \\
\hhline{-}
%row no:5
\multicolumn{1}{|p{7.3in}|}{Notes : n/a} \\
\hhline{-}

\end{tabular}
 \end{table}


%%%%%%%%%%%%%%%%%%%% Table No: 2 ends here %%%%%%%%%%%%%%%%%%%%


\vspace{\baselineskip}

\vspace{\baselineskip}
We already made it work with movies before, now we get bonus points for having it work with other file formats like PDF or MP3. One point per new file format included to work in the media library. They must be functional (meaning playable, or openable). \par


\vspace{\baselineskip}

\vspace{\baselineskip}

\printbibliography
\end{document}