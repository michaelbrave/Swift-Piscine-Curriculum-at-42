%%%%%%%%%%%%  Generated using docx2latex.com  %%%%%%%%%%%%%%

%%%%%%%%%%%%  v2.0.0-beta  %%%%%%%%%%%%%%

\documentclass[12pt]{report}
\usepackage{amsmath}
\usepackage{latexsym}
\usepackage{amsfonts}
\usepackage[normalem]{ulem}
\usepackage{array}
\usepackage{amssymb}
\usepackage{graphicx}
\usepackage[backend=biber,
style=numeric,
sorting=none,
isbn=false,
doi=false,
url=false,
]{biblatex}\addbibresource{bibliography.bib}

\usepackage{subfig}
\usepackage{wrapfig}
\usepackage{wasysym}
\usepackage{enumitem}
\usepackage{adjustbox}
\usepackage{ragged2e}
\usepackage[svgnames,table]{xcolor}
\usepackage{tikz}
\usepackage{longtable}
\usepackage{changepage}
\usepackage{setspace}
\usepackage{hhline}
\usepackage{multicol}
\usepackage{tabto}
\usepackage{float}
\usepackage{multirow}
\usepackage{makecell}
\usepackage{fancyhdr}
\usepackage[toc,page]{appendix}
\usepackage[hidelinks]{hyperref}
\usetikzlibrary{shapes.symbols,shapes.geometric,shadows,arrows.meta}
\tikzset{>={Latex[width=1.5mm,length=2mm]}}
\usepackage{flowchart}\usepackage[paperheight=11.0in,paperwidth=8.5in,left=0.5in,right=0.5in,top=0.5in,bottom=0.5in,headheight=1in]{geometry}
\usepackage[utf8]{inputenc}
\usepackage[T1]{fontenc}
\TabPositions{0.5in,1.0in,1.5in,2.0in,2.5in,3.0in,3.5in,4.0in,4.5in,5.0in,5.5in,6.0in,6.5in,7.0in,}

\urlstyle{same}


 %%%%%%%%%%%%  Set Depths for Sections  %%%%%%%%%%%%%%

% 1) Section
% 1.1) SubSection
% 1.1.1) SubSubSection
% 1.1.1.1) Paragraph
% 1.1.1.1.1) Subparagraph


\setcounter{tocdepth}{5}
\setcounter{secnumdepth}{5}


 %%%%%%%%%%%%  Set Depths for Nested Lists created by \begin{enumerate}  %%%%%%%%%%%%%%


\setlistdepth{9}
\renewlist{enumerate}{enumerate}{9}
		\setlist[enumerate,1]{label=\arabic*)}
		\setlist[enumerate,2]{label=\alph*)}
		\setlist[enumerate,3]{label=(\roman*)}
		\setlist[enumerate,4]{label=(\arabic*)}
		\setlist[enumerate,5]{label=(\Alph*)}
		\setlist[enumerate,6]{label=(\Roman*)}
		\setlist[enumerate,7]{label=\arabic*}
		\setlist[enumerate,8]{label=\alph*}
		\setlist[enumerate,9]{label=\roman*}

\renewlist{itemize}{itemize}{9}
		\setlist[itemize]{label=$\cdot$}
		\setlist[itemize,1]{label=\textbullet}
		\setlist[itemize,2]{label=$\circ$}
		\setlist[itemize,3]{label=$\ast$}
		\setlist[itemize,4]{label=$\dagger$}
		\setlist[itemize,5]{label=$\triangleright$}
		\setlist[itemize,6]{label=$\bigstar$}
		\setlist[itemize,7]{label=$\blacklozenge$}
		\setlist[itemize,8]{label=$\prime$}

\setlength{\topsep}{0pt}\setlength{\parindent}{0pt}

 %%%%%%%%%%%%  This sets linespacing (verticle gap between Lines) Default=1 %%%%%%%%%%%%%%


\renewcommand{\arraystretch}{1.3}


%%%%%%%%%%%%%%%%%%%% Document code starts here %%%%%%%%%%%%%%%%%%%%



\begin{document}

\vspace{\baselineskip}

\vspace{\baselineskip}

\vspace{\baselineskip}

\vspace{\baselineskip}

\vspace{\baselineskip}
\par

\section*{AR Treasure}
\addcontentsline{toc}{section}{AR Treasure}

\vspace{\baselineskip}

\vspace{\baselineskip}
\paragraph*{Michael BRAVE }
\addcontentsline{toc}{paragraph}{Michael BRAVE }
\paragraph*{42 Staff }
\addcontentsline{toc}{paragraph}{42 Staff }

\vspace{\baselineskip}

\vspace{\baselineskip}

\vspace{\baselineskip}

\vspace{\baselineskip}
\begin{Center}
\textit{Summary: This document contains the subject for Day for the $``$Piscine Swift$"$  from 42}
\end{Center}\par


\vspace{\baselineskip}

\vspace{\baselineskip}

\vspace{\baselineskip}

\vspace{\baselineskip}

\vspace{\baselineskip}

\vspace{\baselineskip}

\vspace{\baselineskip}

\vspace{\baselineskip}


 %%%%%%%%%%%%  Starting New Page here %%%%%%%%%%%%%%

\newpage

\vspace{\baselineskip}
\vspace{\baselineskip}
\section*{Contents}
\addcontentsline{toc}{section}{Contents}

\vspace{\baselineskip}
\subsubsection*{I\hspace*{10pt}\hspace*{10pt}Foreword}
\addcontentsline{toc}{subsubsection}{I\hspace*{10pt}\hspace*{10pt}Foreword}
\subsubsection*{II\hspace*{10pt}\hspace*{10pt}General Instructions}
\addcontentsline{toc}{subsubsection}{II\hspace*{10pt}\hspace*{10pt}General Instructions}
\subsubsection*{III\hspace*{10pt}\hspace*{10pt}Introduction}
\addcontentsline{toc}{subsubsection}{III\hspace*{10pt}\hspace*{10pt}Introduction}
\subsubsection*{IV\hspace*{10pt}\hspace*{10pt}Exercise 00: AR Treasure}
\addcontentsline{toc}{subsubsection}{IV\hspace*{10pt}\hspace*{10pt}Exercise 00: AR Treasure}
\subsubsection*{V\hspace*{10pt}\hspace*{10pt}Bonus: }
\addcontentsline{toc}{subsubsection}{V\hspace*{10pt}\hspace*{10pt}Bonus: }

\vspace{\baselineskip}

\vspace{\baselineskip}

\vspace{\baselineskip}

\vspace{\baselineskip}

\vspace{\baselineskip}

\vspace{\baselineskip}


 %%%%%%%%%%%%  Starting New Page here %%%%%%%%%%%%%%

\newpage

\vspace{\baselineskip}
\vspace{\baselineskip}
\section*{Chapter I}
\addcontentsline{toc}{section}{Chapter I}
\section*{Foreword}
\addcontentsline{toc}{section}{Foreword}

\vspace{\baselineskip}
$``$Because, wherever your heart is, that is where you´ll find your treasure$"$  \par

― Paulo Coelho, The Alchemist\par


\vspace{\baselineskip}

\vspace{\baselineskip}

\vspace{\baselineskip}


 %%%%%%%%%%%%  Starting New Page here %%%%%%%%%%%%%%

\newpage

\vspace{\baselineskip}
\vspace{\baselineskip}
\section*{Chapter II}
\addcontentsline{toc}{section}{Chapter II}
\section*{General Instructions}
\addcontentsline{toc}{section}{General Instructions}
\begin{itemize}
	\item Only this document will serve as reference. Do not trust rumors.\par

	\item Read carefully the whole subject before beginning.\par

	\item Watch out! This document could potentially change up to an hour before submission.\par

	\item This project will be corrected by humans only.\par

	\item This course is designed to build on previous days’ concepts, try your hardest to finish everyday.\par

	\item Each day culminates in a portfolio piece, if you finish the day this is something you can use to get hired.\par

	\item When submitting, submit the folder of the Xcode project.\par

	\item Only the work submitted on the repository will be accounted for during peer-2-peer correction.\par

	\item Here it is the \href{https://docs.swift.org/swift-book/}{\textcolor[HTML]{1155CC}{\uline{official manual of Swift}}} and the \href{https://developer.apple.com/documentation/swift/swift_standard_library}{\textcolor[HTML]{1155CC}{\uline{Swift Standard Library}}}\par

	\item It is forbidden to use other libraries, packages, pods, etc. Unless otherwise stated in the project.\par

	\item Got a question? Ask your peer on the right. Otherwise, try your peer on the left.\par

	\item You can discuss on the Piscine forum of your Intra!\par

	\item By Odin, by Thor! Use your brain!!!
\end{itemize}\par


\vspace{\baselineskip}

\vspace{\baselineskip}


 %%%%%%%%%%%%  Starting New Page here %%%%%%%%%%%%%%

\newpage

\vspace{\baselineskip}
\vspace{\baselineskip}
\section*{Chapter III}
\addcontentsline{toc}{section}{Chapter III}
\section*{Introduction}
\addcontentsline{toc}{section}{Introduction}
This project will require using a test device, if you don’t have a test device then find a partner who does, if unable to find one do rush 01 (alternate) which doesn’t require a test device. \par


\vspace{\baselineskip}
Today we are building an Augmented Reality (AR) app in which we will find hidden treasures. Functionally this is similar to pokemon go, but instead of adorable pokemon we will be collecting treasure. \par


\vspace{\baselineskip}


 %%%%%%%%%%%%  Starting New Page here %%%%%%%%%%%%%%

\newpage

\vspace{\baselineskip}
\vspace{\baselineskip}
\section*{Chapter IV}
\addcontentsline{toc}{section}{Chapter IV}
\section*{Exercise 00 : AR Treasure}
\addcontentsline{toc}{section}{Exercise 00 : AR Treasure}

\vspace{\baselineskip}

\vspace{\baselineskip}

\vspace{\baselineskip}


%%%%%%%%%%%%%%%%%%%% Table No: 1 starts here %%%%%%%%%%%%%%%%%%%%


\begin{table}[H]
 			\centering
\begin{tabular}{p{7.3in}}
\hline
%row no:1
\multicolumn{1}{|p{7.3in}|}{\Centering Exercise : 00} \\
\hhline{-}
%row no:2
\multicolumn{1}{|p{7.3in}|}{\Centering AR Treasure} \\
\hhline{-}
%row no:3
\multicolumn{1}{|p{7.3in}|}{Files to turn in: .xcodeproj and all necessary files} \\
\hhline{-}
%row no:4
\multicolumn{1}{|p{7.3in}|}{Allowed functions : Swift Standard Library, UIKit} \\
\hhline{-}
%row no:5
\multicolumn{1}{|p{7.3in}|}{Notes : n/a} \\
\hhline{-}

\end{tabular}
 \end{table}


%%%%%%%%%%%%%%%%%%%% Table No: 1 ends here %%%%%%%%%%%%%%%%%%%%


\vspace{\baselineskip}

\vspace{\baselineskip}
We are creating an augmented reality app similar in function to pokemon go. It will use location, cameras, and digital overlays to populate the screen with treasure that we will collect by clicking on the screen. This will be overlayed on top of and interact with what is shown in camera. \par


\vspace{\baselineskip}
When we find a treasure chest we will have to swipe at it to weaken it, then tap it to open it, the chest will open and give us treasure that will count toward our overall score. There should be at least 4 types of treasure chests that spawn at random. \par


\vspace{\baselineskip}


 %%%%%%%%%%%%  Starting New Page here %%%%%%%%%%%%%%

\newpage

\vspace{\baselineskip}
\vspace{\baselineskip}
\section*{Chapter V}
\addcontentsline{toc}{section}{Chapter V}
\section*{Bonus : Use Locations To Hide Treasures}
\addcontentsline{toc}{section}{Bonus : Use Locations To Hide Treasures}

\vspace{\baselineskip}

\vspace{\baselineskip}

\vspace{\baselineskip}


%%%%%%%%%%%%%%%%%%%% Table No: 2 starts here %%%%%%%%%%%%%%%%%%%%


\begin{table}[H]
 			\centering
\begin{tabular}{p{7.3in}}
\hline
%row no:1
\multicolumn{1}{|p{7.3in}|}{\Centering Bonus} \\
\hhline{-}
%row no:2
\multicolumn{1}{|p{7.3in}|}{\Centering Use Locations To Hide Treasures} \\
\hhline{-}
%row no:3
\multicolumn{1}{|p{7.3in}|}{Files to turn in: .xcodeproj and all necessary files} \\
\hhline{-}
%row no:4
\multicolumn{1}{|p{7.3in}|}{Allowed functions : Swift Standard Library, UIKit} \\
\hhline{-}
%row no:5
\multicolumn{1}{|p{7.3in}|}{Notes : n/a} \\
\hhline{-}

\end{tabular}
 \end{table}


%%%%%%%%%%%%%%%%%%%% Table No: 2 ends here %%%%%%%%%%%%%%%%%%%%


\vspace{\baselineskip}
Use specific pre built locations to hide extra special otherwise inaccessible treasures that can only be found while in a specific predefined location. \par


\vspace{\baselineskip}

\vspace{\baselineskip}

\printbibliography
\end{document}